\documentclass[11pt]{article}

\usepackage[fontset=none]{ctex}
\setmainfont[Extension={.otf},
    UprightFont={*-Regular},
    BoldFont={*-Bold},
    ItalicFont={*-Italic},
    BoldItalicFont={*-BoldItalic}]{STIXTwoText}
\setCJKmainfont[BoldFont={FZCuSong-B09S},ItalicFont={FZKai-Z03S},BoldItalicFont={FZCuKaiS-R-GB}]{FZShuSong-Z01S}
\setCJKsansfont{FZHei-B01S}
\setCJKmonofont{FZFangSong-Z02S}
\usepackage[paper=a4paper]{geometry}
\usepackage{amsmath}
\usepackage{amssymb}
\usepackage{amsthm}
\usepackage{autobreak}
\usepackage{unicode-math}
\setmathfont{STIX Two Math}
\setmathfont{TeX Gyre Pagella Math}[range={bb,frak}]
\usepackage[shortlabels]{enumitem}
\setlist{nosep}
\usepackage{tikz-cd}
\usepackage{biblatex}
\addbibresource{biblio.bib}
\usepackage{hyperref}
\usepackage{fixdif}

\renewcommand{\proofname}{{\bf 证明}}
\theoremstyle{definition}\newtheorem*{analyse}{分析}
\theoremstyle{remark}\newtheorem{rem*}{评注}

%\def\d#1{\ \mathrm{d}#1}
\newenvironment{env}[1]{\par\vspace{1em}\noindent\textbf{#1}\ }{\par\vspace{1em}}
\DeclareMathOperator{\supp}{supp}
\DeclareMathOperator{\im}{im}

\title{de Rham上同调与Jordan曲线定理}
\author{魏子涵}
\date{\today}

\begin{document}
\maketitle

\setcounter{section}{-1}
\section{微分形式回顾}
我们回顾一下需要用到的有关微分形式的知识.

一个\textbf{$1$--形式}是一个形式和$p\d{x}+q\d{y}$, 其中$p,q$是光滑函数.
一个光滑函数$f$的\textbf{微分}是一个$1$--形式$(\partial f/\partial x)\d{x}+(\partial f/\partial y)\d{y}$.
$1$--形式中的\textbf{闭形式}是指满足$\partial p/\partial x=\partial q/\partial y$的$p\d{x}+q\d{y}$;
\textbf{恰当形式}是指作为一个光滑函数微分的$1$--形式.
闭形式与恰当形式均构成实向量空间.

一个简单的结论是
\begin{env}{准则1.10}
    恰当形式都是闭形式. \cite[p.\ 10]{Fulton1995}
\end{env}

以及一个部分的逆命题
\begin{env}{命题1.12}
    设$U$是两个有限或无限长的开区间的乘积, 那么在$U$上闭形式都是恰当形式.
\end{env}

我们之后需要一个更强的命题
\begin{env}{Poincar\'{e}引理}
    如果$U$中存在一点$P$, 使得任意$Q\in U$与$P$连接的线段均在$U$内 (即$U$是\textit{星形区域}), 那么$U$中的闭形式都是恰当形式.
\end{env}

证明参阅~\cite[定理4.11]{Spivak1965}.
此外, 我们需要一个关于卷绕数的命题
\begin{env}{命题3.16}
    设$\gamma$是闭路径, $\supp\gamma$是$\gamma$的图像,
    那么$\gamma$的卷绕数 (作为点的函数) 在$\mathbb{R}^2\backslash\supp\gamma$的每个连通分支上是常值函数.
    特别地, 在无界的连通分支上卷绕数为$0$.
\end{env}

\section{de Rham上同调群}
本书在这一章只考虑平面上一个开集$U$的第零阶和第一阶上同调群, 分别定义为
\begin{gather*}
    H^0(U)=\{f\in C^\infty(U)|\ f\ \text{局部为常值}\}\\
    H^1(U)=\frac{\{U\text{上的闭}1\text{--形式}\}}{\{U\text{上的恰当}1\text{--形式}\}}
\end{gather*}
这两者实质上是实向量空间, 但在传统上会把上同调叫做群.

容易看出$\dim H^0(U)$就是$U$的连通分支个数.
而对于$H^1$, 这一节有一个简单的计算 (命题5.1), 我们直接证明问题5.2的推广形式:
\begin{env}{问题5.2}
    设$X=\{P_1,\cdots,P_n\}$是$\mathbb{R}^2$上$n$个点构成的集合, 那么$H^1(\mathbb{R}^2\backslash X)\cong\mathbb{R}^n$.
\end{env}
\begin{analyse}
    本书是通过直接给出生成元来证明这个命题的.
    对$P=(x_0,y_0)\in\mathbb{R}^2$与某个含$P$的开集$U$, 设$1$--形式
    \[\omega_P:=\frac{1}{2\pi}\frac{-(y-y_0)\d{x}+(x-x_0)\d{y}}{(x-x_0)^2+(y-y_0)^2}\]
    那么$\omega_P$都是闭形式, 记$[\omega_P]$为$\omega_P$在$H^1(U\backslash\{P\})$中的等价类.
    我们希望证明$n$个形式的等价类$[\omega_{P_1}],\cdots,[\omega_{P_n}]$是$H^1(\mathbb{R}^2\backslash X)$的一组基.
    这需要用到引理1.17到问题1.19的一个结论:
    设$r<\min_{i,j}\{|P_iP_j|\}$, $C_i$为以$P_i$为圆心半径为$r$的圆, 如果$1$--形式$\omega$满足对每个$i=1,2,\cdots,n$都有
    \begin{equation}
        \int_{C_i}\omega=0\label{prob1.19}
    \end{equation}
    那么$\omega$在$\mathbb{R}^2\backslash X$上是恰当的.
    证明这个结论通过归纳法, 通过在一个点处画两根直线, 将平面分为$4$个半平面, 然后在这四个半平面上均存在光滑函数使其微分为$\omega$,
    积分为$0$保证可以通过调整常数使得这些函数在半平面的重合处相等, 从而$\omega$使恰当的.
\end{analyse}
\begin{env}{问题1.19}
    设$U$是两个有限或无限长的开区间的乘积, $X$是$U$中有限个点的集合, 那么满足~\eqref{prob1.19}~式假设的$1$--形式$\omega$是恰当的.
\end{env}
\begin{proof}
    我们对$X$中的点的个数$n$使用归纳法.
    $n=0$时是命题1.12.
    假设命题对所有不超过$n$个点的情况成立.
    对$n+1$个点构成的集合$X$, 取$X$中的一点$P$, 在$P$处平行于$U$的边建立坐标系.
    设$U$与第I, II象限及$y$轴正半轴的交为$U_1$, 类似地逆时针旋转定义$U_2,U_3,U_4$.
    那么每个$U_i$均为去掉不超过$n$个点的两个开区间的乘积, 从而按归纳假设, 存在$U_i$上的光滑函数$f_i$满足$\d{f_i}=\omega$.
    由于在重叠处有$\d{f_i}=\d{f_j}$, 所以$f_i$与$f_j$在$U_i\cap U_j$上只相差一个常数 (因重叠处是连通的).
    那么可以通过调整常数使得$f_1=f_2,f_2=f_3,f_3=f_4$.
    取$Q\in U_1\cap U_4$, 考虑
    \[\int_{C_{P}}\omega=f_4(Q)-f_1(Q)=0\]
    从而有$f_1=f_4$, 因此在$U_1\cup\cdots\cup U_4=U\backslash X$上存在$f$使得$\d{f}=\omega$.
\end{proof}
\begin{proof}[{\bf 问题5.2的证明}]
    我们首先证明$[\omega_{P_1}],\cdots,[\omega_{P_n}]$是线性无关的.
    设
    \[\sum_{i=1}^na_i[\omega_{P_i}]=0\]
    即$\sum_{i=1}^na_i\omega_{P_i}$是恰当的, 那么有
    \[0=\int_{C_i}\sum_{i=1}^na_i\omega_{P_i}=a_i\]
    从而$[\omega_{P_1}],\cdots,[\omega_{P_n}]$线性无关.
    我们现在证明$[\omega_{P_1}],\cdots,[\omega_{P_n}]$生成$H^1(\mathbb{R}^2\backslash X)$.
    设$\omega_0$是闭的$1$--形式, 记
    \[\int_{C_i}\omega_0=c_i\]
    设$\omega:=\omega_0-\sum_{i=1}^nc_i\omega_{P_i}$, 由于对每个$C_i$都有
    \[\int_{C_i}\omega=0\]
    所以$\omega$在$\mathbb{R}^2\backslash X$上是恰当的.
    因此就有$[\omega_0]=\sum_{i=0}^nc_i[\omega_{P_i}]$, 从而
    \[H^1(\mathbb{R}^2\backslash X)=\left\langle[\omega_{P_1}],\cdots,[\omega_{P_n}]\right\rangle\cong\mathbb{R}^n\]
\end{proof}

本章还会用到一个结论
\begin{env}{命题5.3}
    设$A$是$\mathbb{R}^2$上的连通闭集, $P,Q\in A$, 那么在$H^1(\mathbb{R}^2\backslash A)$中有$[\omega_P]=[\omega_Q]$.
\end{env}
\begin{proof}
    我们只需要证明$\omega_P-\omega_Q$在$\mathbb{R}^2\backslash A$中任意闭路径上的积分为$0$.
    而对闭路径$\gamma$有
    \[\int_{\gamma}(\omega_P-\omega_Q)=W(\gamma,P)-W(\gamma,Q)\]
    按命题3.16, $\gamma$关于$P$和$Q$的卷绕数是相等的, 所以以上积分为$0$, 从而命题得证.
\end{proof}

\section{上边缘映射}
我们换用一个更明晰一些的方法来讲这一节的内容.

对两个开集$U$与$V$, 考虑嵌入映射构成的交换图
\[\begin{tikzcd}
    & U \arrow[dr, "k"] & \\
    U\cap V \arrow[ur, "i"] \arrow[dr, "j"'] & & U\cup V\\
    & V \arrow[ur, "l"'] &
\end{tikzcd}\]
设线性映射
\begin{gather*}
    i^*:H^0(U)\to H^0(U\cap V),\ f\mapsto f|_{U\cap V}\\
    j^*:H^0(V)\to H^0(U\cap V),\ g\mapsto g|_{U\cap V}\\
    k^*:H^1(U\cap V)\to H^1(U),\ [\omega]\mapsto[\omega|_U]\\
    l^*:H^1(U\cap V)\to H^1(V),\ [\tau]\mapsto[\tau|_V]
\end{gather*}
那么定义\textbf{上边缘映射}为一个线性映射$\delta:H^0(U\cap V)\to H^1(U\cup V)$, 满足序列
\begin{equation}
    H^0(U)\oplus H^0(V)\xrightarrow{i^*-j^*}H^0(U\cap V)\xrightarrow{\delta}H^1(U\cup V)\xrightarrow{k^*\oplus l^*}H^1(U)\oplus H^1(V)
    \label{mayer-vietoris}
\end{equation}
是正合的.
按照这个定义, $\delta$满足
\begin{enumerate}
    \item 对$f\in\ker\delta$, 存在$U$和$V$上的局部常值函数$f_1,f_2$使得$f=f_1-f_2$ (严格写应该是$f_1|_{U\cap V}-f_2|_{U\cap V}$, 但书上的命题5.7没有这么写);
    \item 如果$\omega\in\im\delta$, 那么$\omega$在$U$和$V$上的限制都是恰当的 (命题5.9前半部分);
    \item 如果$H^1(U)=H^1(V)=0$, 那么$\delta$是满射 (命题5.9后半部分).
\end{enumerate}

我们证明上边缘映射的存在性.
这需要用到\textit{单位分解定理}.
(证明见~\cite[附录B2]{Fulton1995}~或~\cite[pp.\ 63--64]{Spivak1965})

\begin{proof}
    取$U\cup V$的从属于$\{U,V\}$的一组单位分解$\varphi_1,\varphi_2$, 即满足$\supp\varphi_1\subset U,\supp\varphi_2\subset V$且$\varphi_1+\varphi_2=1$.
    对$f\in H^0(U\cap V)$, 取
    \begin{gather*}
        f_1(p)=\begin{cases}
            \varphi_2(p)f(p), & p\in U\cap V\\
            0, & p\in U\backslash V
        \end{cases}\\
        f_2(p)=\begin{cases}
            -\varphi_1(p)f(p), & p\in U\cap V\\
            0, & p\in V\backslash U
        \end{cases}
    \end{gather*}
    那么$f_1,f_2$都是光滑的, 且$f_2|_{U\cap V}-f_1|_{U\cap V}=f$.
    由于$f$是局部常值的, 那么有$0=\d{f}=\d{f_2|_{U\cap V}}-\d{f_1|_{U\cap V}}$, 从而存在闭形式$\omega_f\in U\cup V$使得在$U\cap V$上$\omega_f=\d{f_1}=\d{f_2}$.
    定义$\delta(f)=[\omega_f]$.
    我们说明$\delta$是良定义的.
    如果另外有$f_3|_{U\cap V}-f_4|_{U\cap V}=f$与由$f_3,f_4$定义的$\omega_f'$, 那么在$U\cap V$上有
    \[f_1-f_2=f_3-f_4\implies f_1-f_3=f_2-f_4\]
    从而可以在$U\cup V$上定义$f'$满足$f'|_{U\cap V}=f_1-f_3=f_2-f_4$, 且有$\d{f'}=\omega_f-\omega_f'$.
    那么$\omega_f-\omega_f'$是恰当的, 有$[\omega_f]=[\omega_f']$, 因此$\delta$是良定义的.
    以下我们验证序列~\eqref{mayer-vietoris}~是正合的.
    首先证明$\im(i^*-j^*)=\ker\delta$.
    如果$f=g|_{U\cap V}-h|_{U\cap V}$, 其中$g,h$分别是$V,U$上的局部常值函数, 那么一定有$\d{g}=0,\d{h}=0$, 从而$\delta{f}=0$.
    反过来, 如果$\delta{f}=0$, 按以上过程取$f_2-f_1=f$, 并设$\omega_f|_U=\d{f_1},\omega_f|_V=\d{f_2}$.
    由于$\delta{f}=[\omega_f]=0$, 可知$\omega_f$是恰当的, 设$\omega_f=\d{g}$, 那么有
    \[\d{f_1}-\d{g|_U}=0,\d{f_2}-\d{g|_V}=0\]
    所以$g-f_1$与$g-f_2$是局部常值函数, 满足$(i^*-j^*)(g-f_1,g-f_2)=f$.
    其次证明$\im\delta=\ker{k^*\oplus l^*}$.
    如果$f=f_2-f_1$, 那么
    \[k^*(\delta(f))=\d{(\d{f_1})}=0,l^*(\delta(f))=\d{(\d{f_2})}=0\]
    从而$\im\delta\subset\ker(k^*\oplus l^*)$.
    反过来, 对任意$[\omega]\in\ker(k^*\oplus l^*)$, $k^*\oplus l^*([\omega])=0$意味着$\omega|_U$与$\omega|_V$都是恰当的, 从而存在$U,V$上的光滑函数$f_1,f_2$使得$\omega|_U=\d{f_1},\omega|_V=\d{f_2}$.
    定义$U\cap V$上的光滑函数$f=f_2-f_1$, 那么$\d{f}=\d{f_2}-\d{f_1}=0$, 从而$f$是局部常值的, 即$f\in H^0(U\cap V)$, 那么按定义就有$\delta(f)=[\omega]$, 所以$\im\delta=\ker{k^*\oplus l^*}$.
\end{proof}

{\it 上边缘映射本质上是\textbf{Mayer-Vietoris序列}的一部分, 在第10章与第23, 24章会讲到这个.}

\section{Jordan曲线定理}
这一节要证明
\begin{env}{定理5.10}(Jordan曲线定理)
    如果$X\subset\mathbb{R}^2$与一个圆周同胚, 那么$\mathbb{R}^2\backslash X$有两个连通分支, 一个有界, 另一个无界.
    $X$上任意一点的任意邻域均同时包含这两个两个连通分支中的点.
\end{env}
\noindent 而这依赖于一个命题
\begin{env}{定理5.11}
    如果$Y\subset\mathbb{R}^2$同胚于一个闭区间, 那么$\mathbb{R}^2\backslash Y$是连通的.
\end{env}
\noindent 这两个命题都需要代数拓扑的方法来证明.

我们先证明定理5.11.
\begin{proof}
    设$I=[0,1]$是单位闭区间, $f:I\to Y$是同胚.
    反设$\mathbb{R}^2\backslash Y$是不连通的, 即
    \[\dim H^0(\mathbb{R}^2\backslash Y)\geq 2\]
    取定两个在不同连通分支中的点$A,B$.
    设$Y$中与$\left[0,\frac{1}{2}\right]$同胚的部分为$Y^+$, 与$\left[\frac{1}{2},1\right]$同胚的部分为$Y^-$, $Y^+\cap Y^-=Q$.
    考虑Mayer-Vietoris序列
    \begin{align*}
    H^0(\mathbb{R}^2\backslash Y^+)\oplus H^0(\mathbb{R}^2\backslash Y^-)&\longrightarrow H^0(\mathbb{R}^2\backslash Y)
    \longrightarrow H^1(\mathbb{R}^2\backslash\{Q\})\\
    &\longrightarrow H^1(\mathbb{R}^2\backslash Y^+)\oplus H^1(\mathbb{R}^2\backslash Y^-)
    \end{align*}
    对$k^*\oplus l^*$而言, 由于$H^1(\mathbb{R}^2\backslash\{Q\})=\left\langle[\omega_{Q}]\right\rangle$, 而$\omega_{Q}|_{\mathbb{R}^2\backslash Y^+}$与$\omega_{Q}|_{\mathbb{R}^2\backslash Y^-}$均为非零的闭形式, 所以$k^*\oplus l^*$是单射.
    因此有$\im\delta=\ker(k^*\oplus l^*)=0$, 那么正合列
    \[H^0(\mathbb{R}^2\backslash Y^+)\oplus H^0(\mathbb{R}^2\backslash Y^-)\xrightarrow{i^*-j^*} H^0(\mathbb{R}^2\backslash Y)\xrightarrow{\delta}0\]
    说明$i^*-j^*$是满射. 对某个在$A,B$处取值不同的$f\in H^0(\mathbb{R}^2\backslash Y)$, 存在$g\in H^0(\mathbb{R}^2\backslash Y^+),h\in H^0(\mathbb{R}^2\backslash Y^-)$使得$g|_{\mathbb{R}^2\backslash Y}-h|_{\mathbb{R}^2\backslash Y}=f$.
    如果$g,h$均在$A,B$处取值相等, 可以推出$f$在$A,B$处取值相等, 矛盾.
    所以$A,B$在$H^0(\mathbb{R}^2\backslash Y^+)$或$H^0(\mathbb{R}^2\backslash Y^+)$的两个不同连通分支中, 设满足的那个为$\mathbb{R}^2\backslash Y_1$.
    依次构造下去, 我们可以得到一个集合列
    \[Y=Y_0\supset Y_1\supset\cdots\supset Y_n\supset\cdots\]
    设$\bigcap_{n=1}^\infty Y_n=\{P\}$.
    对$\mathbb{R}^2\backslash Y$任意两个不同连通分支中的点$A,B$, 由于$\mathbb{R}^2\backslash\{P\}$是连通的, 存在一条闭路径$\gamma$连接$A,B$.
    那么存在$P$的一个邻域$N$与$\gamma$不交, 但又存在充分大的$n$使得$Y_n\subset N$, 从而$\supp\gamma\subset\mathbb{R}^2\backslash Y_n$, 矛盾.
\end{proof}

\begin{proof}[{\bf Jordan曲线定理的证明}]
    设$P,Q$是$X$上任意两点, 连接$P,Q$的两段闭弧记为$A,B$.
    考虑对$U=\mathbb{R}^2\backslash A,V=\mathbb{R}^2\backslash B$的Mayer-Vietoris序列
    \begin{align*}
        H^0(\mathbb{R}^2\backslash A)\oplus H^0(\mathbb{R}^2\backslash B)&\xrightarrow{i^*-j^*}H^0(\mathbb{R}^2\backslash X)\xrightarrow{\delta}H^1(\mathbb{R}^2\backslash\{P,Q\})\\
        &\xrightarrow{k^*\oplus l^*}H^1(\mathbb{R}^2\backslash A)\oplus H^1(\mathbb{R}^2\backslash B)
    \end{align*}
    由定理5.11, $\mathbb{R}^2\backslash A$与$\mathbb{R}^2\backslash B$是连通的, 所以$H^0(\mathbb{R}^2\backslash A)$与$H^0(\mathbb{R}^2\backslash B)$由常值函数构成.
    对$\mathbb{R}^2\backslash A$与$\mathbb{R}^2\backslash B$上的常值函数$g,h$, 那么有$(i^*-j^*)(g,h)=g|_{\mathbb{R}^2\backslash X}-h|_{\mathbb{R}^2\backslash X}$为常值函数, 从而$\dim\im(i^*-j^*)=1$.
    由正合可知$\dim\ker\delta=\dim\im(i^*-j^*)=1$.
    而对于$k^*\oplus l^*$, 由于$H^1(\mathbb{R}^2\backslash\{P,Q\})=\left\langle[\omega_P],[\omega_Q]\right\rangle$, 而由命题5.3知在$\mathbb{R}^2\backslash A$上有$[\omega_P]=[\omega_Q]$, 在$\mathbb{R}^2\backslash B$上同理, 所以有$k^*\oplus l^*([\omega|_P])=k^*\oplus l^*([\omega|_Q])$.
    因此$\dim\im k^*\oplus l^*=1$, 而$\dim H^1(\mathbb{R}^2\backslash\{P,Q\})=2$, 有
    \begin{align*}
        \dim\im\delta&=\dim\ker k^*\oplus l^*\\
        &=\dim H^1(\mathbb{R}^2\backslash\{P,Q\})-\dim\im k^*\oplus l^*\\
        &=1
    \end{align*}
    所以$\dim H^0(\mathbb{R}^2\backslash X)=\dim\ker\delta+\dim\im\delta=2$, 即$\mathbb{R}^2\backslash X$有两个连通分支, 显然一个有界而另一个无界.

    我们证明最后一个断言.
    对任意一点$T\in X$及$T$的邻域$N$, 在$N$中取两点$P,Q$, 设两点将$X$分的两段闭弧为$A,B$, 其中$A$在$N$内.
    设$P_0,P_1$是$\mathbb{R}^2\backslash X$的两个不同连通分支中的点, 由于$\mathbb{R}^2\backslash B$是连通的, 所以存在闭路径$\gamma$连接这两点.
    $\gamma$一定与$A$相交 (否则$X$将会连通), 且交点的集合是闭集, 那么设
    \[t_m=\min_{t\in[0,1]}\{t|\ \gamma(t)\in A\},t_M=\max_{t\in[0,1]}\{t|\ \gamma(t)\in A\}\]
    那么存在某个充分小的$\varepsilon$, 使得$\gamma(t_m-\varepsilon)$与$P_0$在同一连通分支而$\gamma(t_M+\varepsilon)$与$P_1$在同一连通分支, 且$\gamma(t_m-\varepsilon),\gamma(t_M+\varepsilon)$均在$N$中.
    从而命题得证.
\end{proof}

\printbibliography[title={参考文献}]
\end{document}