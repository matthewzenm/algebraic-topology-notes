\documentclass[11pt]{article}

\usepackage[fontset=none]{ctex}
\setmainfont{XITS}
\setCJKmainfont[BoldFont={FZCuSong-B09S},ItalicFont={FZKai-Z03S},BoldItalicFont={FZCuKaiS-R-GB}]{FZShuSong-Z01S}
\setCJKsansfont{FZHei-B01S}
\setCJKmonofont{FZFangSong-Z02S}
\usepackage{amsmath}
\usepackage{amssymb}
\usepackage{amsthm}
\usepackage{unicode-math}
\setmathfont{XITS Math}
\setmathfont{TeX Gyre Pagella Math}[range={bb,frak}]
\usepackage[shortlabels]{enumitem}
\setlist{nosep}
\usepackage{tikz-cd}
\usepackage{biblatex}
\addbibresource{biblio.bib}
\usepackage{hyperref}

\renewcommand{\proofname}{{\bf 证明}}
\theoremstyle{definition}\newtheorem*{analyse}{分析}

\def\d#1{\ \mathrm{d}#1}
\newenvironment{env}[1]{\par\vspace{1em}\noindent\textbf{#1}\ }{\par\vspace{1em}}
\DeclareMathOperator{\supp}{supp}
\DeclareMathOperator{\im}{im}

\title{de Rham上同调与Jordan曲线定理}
\author{魏子涵}
\date{\today}

\begin{document}
\maketitle

\setcounter{section}{-1}
\section{微分形式回顾}
我们回顾一下需要用到的有关微分形式的知识.

一个\textbf{$1$--形式}是一个形式和$p\d{x}+q\d{y}$, 其中$p,q$是光滑函数.
一个光滑函数$f$的\textbf{微分}是一个$1$--形式$(\partial f/\partial x)\d{x}+(\partial f/\partial y)\d{y}$.
$1$--形式中的\textbf{闭形式}是指满足$\partial p/\partial x=\partial q/\partial y$的$p\d{x}+q\d{y}$;
\textbf{恰当形式}是指作为一个光滑函数微分的$1$--形式.
闭形式与恰当形式均构成实向量空间.

一个简单的结论是
\begin{env}{准则1.10}
    恰当形式都是闭形式. \cite[p.\ 10]{Fulton1995}
\end{env}

以及一个部分的逆命题
\begin{env}{命题1.12}
    设$U$是两个有限或无限长的开区间的乘积, 那么在$U$上闭形式都是恰当形式.
\end{env}

我们之后需要一个更强的命题
\begin{env}{Poincar\'{e}引理}
    如果$U$中存在一点$P$, 使得任意$Q\in U$与$P$连接的线段均在$U$内 (即$U$是\textit{星形区域}), 那么$U$中的闭形式都是恰当形式.
\end{env}

证明参阅~\cite[定理4.11]{Spivak1965}.
此外, 我们需要一个关于卷绕数的命题
\begin{env}{命题3.16}
    设$\gamma$是闭路径, $\supp\gamma$是$\gamma$的图像,
    那么$\gamma$的卷绕数 (作为点的函数) 在$\mathbb{R}^2\backslash\supp\gamma$的每个连通分支上是常值函数.
    特别地, 在无界的连通分支上卷绕数为$0$.
\end{env}

\section{de Rham上同调群}
本书在这一章只考虑平面上一个开集$U$的第零阶和第一阶上同调群, 分别定义为
\begin{gather*}
    H^0(U)=\{f\in C^\infty(U)|\ f\ \text{局部为常值}\}\\
    H^1(U)=\frac{\{U\text{上的闭}1\text{--形式}\}}{\{U\text{上的恰当}1\text{--形式}\}}
\end{gather*}
这两者实质上是实向量空间, 但在传统上会把上同调叫做群.

容易看出$\dim H^0(U)$就是$U$的连通分支个数.
而对于$H^1$, 这一节有一个简单的计算 (命题5.1), 我们直接证明问题5.2的推广形式:
\begin{env}{问题5.2}
    设$X=\{P_1,\cdots,P_n\}$是$\mathbb{R}^2$上$n$个点构成的集合, 那么$H^1(\mathbb{R}^2\backslash X)\cong\mathbb{R}^n$.
\end{env}
\begin{analyse}
    本书是通过直接给出生成元来证明这个命题的.
    对$P=(x_0,y_0)\in\mathbb{R}^2$与某个含$P$的开集$U$, 设$1$--形式
    \[\omega_P:=\frac{1}{2\pi}\frac{-(y-y_0)\d{x}+(x-x_0)\d{y}}{(x-x_0)^2+(y-y_0)^2}\]
    那么$\omega_P$都是闭形式, 记$[\omega_P]$为$\omega_P$在$H^1(U\backslash\{P\})$中的等价类.
    我们希望证明$[\omega_{P_1}],\cdots,[\omega_{P_n}]$是$H^1(\mathbb{R}^2\backslash X)$的一组基.
    这需要用到引理1.17到问题1.19的一个结论:
    设$r<\min_{i,j}\{|P_iP_j|\}$, $C_i$为以$P_i$为圆心半径为$r$的圆, 如果$1$--形式$\omega$满足对每个$i=1,2,\cdots,n$都有
    \begin{equation}
        \int_{C_i}\omega=0\label{prob1.19}
    \end{equation}
    那么$\omega$在$\mathbb{R}^2\backslash X$上是恰当的.
    证明这个结论通过归纳法, 通过在一个点处画两根直线, 将平面分为$4$个半平面, 然后在这四个半平面上均存在光滑函数使其微分为$\omega$,
    积分为$0$保证可以通过调整常数使得这些函数在半平面的重合处相等, 从而$\omega$使恰当的.
\end{analyse}
\begin{env}{问题1.19}
    设$U$是两个有限或无限长的开区间的乘积, $X$是$U$中有限个点的集合, 那么满足~\eqref{prob1.19}~假设的$1$--形式$\omega$是恰当的.
\end{env}
\begin{proof}
    未完待续.
\end{proof}
\begin{proof}[{\bf 问题5.2的证明}]
    我们首先证明$[\omega_{P_1}],\cdots,[\omega_{P_n}]$是线性无关的.
    设
    \[\sum_{i=1}^na_i[\omega_{P_i}]=0\]
    即$\sum_{i=1}^na_i\omega_{P_i}$是恰当的, 那么有
    \[0=\int_{C_i}\sum_{i=1}^na_i\omega_{P_i}=a_i\]
    从而$[\omega_{P_1}],\cdots,[\omega_{P_n}]$线性无关.
    我们现在证明$[\omega_{P_1}],\cdots,[\omega_{P_n}]$生成$H^1(\mathbb{R}^2\backslash X)$.
    设$\omega_0$是闭的$1$--形式, 记
    \[\int_{C_i}\omega_0=c_i\]
    我们证明$\omega:=\omega_0-\sum_{i=1}^nc_i\omega_{P_i}$是恰当的.
\end{proof}

本章还会用到一个结论
\begin{env}
    设$A$是$\mathbb{R}^2$上的连通闭集, $P,Q\in A$, 那么在$H^1(\mathbb{R}^2\backslash A)$中有$[\omega_P]=[\omega_Q]$.
\end{env}

\section{上边缘映射}
我们换用一个更明晰一些的方法来讲这一节的内容.

对两个开集$U$与$V$, 考虑嵌入映射构成的交换图
\[\begin{tikzcd}
    & U \arrow[dr, "k"] & \\
    U\cap V \arrow[ur, "i"] \arrow[dr, "j"'] & & U\cup V\\
    & V \arrow[ur, "l"'] &
\end{tikzcd}\]
设线性映射
\begin{gather*}
    i^*:H^0(U)\to H^0(U\cap V),\ f\mapsto f|_{U\cap V}\\
    j^*:H^0(V)\to H^0(U\cap V),\ g\mapsto g|_{U\cap V}\\
    k^*:H^1(U\cap V)\to H^1(U),\ [\omega]\mapsto[\omega|_U]\\
    l^*:H^1(U\cap V)\to H^1(V),\ [\tau]\mapsto[\tau|_V]
\end{gather*}
那么定义\textbf{上边缘映射}为一个线性映射$\delta:H^0(U\cap V)\to H^1(U\cup V)$, 满足序列
\[H^0(U)\oplus H^0(V)\xrightarrow{i^*-j^*}H^0(U\cap V)\xrightarrow{\delta}H^1(U\cup V)\xrightarrow{k^*\oplus l^*}H^1(U)\oplus H^1(V)\]
是正合的.
按照这个定义, $\delta$满足
\begin{enumerate}
    \item 对$f\in\ker\delta$, 存在$U$和$V$上的局部常值函数$f_1,f_2$使得$f=f_1-f_2$ (严格写应该是$f_1|_{U\cap V}-f_2|_{U\cap V}$, 但书上的命题5.7没有这么写);
    \item 如果$\omega\in\im\delta$, 那么$\omega$在$U$和$V$上的限制都是恰当的 (命题5.9前半部分);
    \item 如果$H^1(U)=H^1(V)=0$, 那么$\delta$是满射 (命题5.9后半部分).
\end{enumerate}

我们证明上边缘映射的存在性.
这需要用到\textit{单位分解定理}.

{\it 上边缘映射本质上是\textbf{Mayer-Vietoris序列}的一部分, 在第10章与第23, 24章会讲到这个.}

\section{Jordan曲线定理}
这一节要证明
\begin{env}{定理5.10}(Jordan曲线定理)
    如果$X\subset\mathbb{R}^2$与一个圆周同胚, 那么$\mathbb{R}^2\backslash X$有两个连通分支, 一个有界, 另一个无界.
    $X$上任意一点的任意邻域均同时包含这两个两个连通分支中的点.
\end{env}
\noindent 而这依赖于一个命题
\begin{env}{定理5.11}
    如果$Y\subset\mathbb{R}^2$同胚于一个闭区间, 那么$\mathbb{R}^2\backslash Y$是连通的.
\end{env}
\noindent 这两个命题都需要代数拓扑的方法来证明.

我们先证明定理5.11.

\section{应用和变体}

本节需要用到一个 (现在还证不出来的) 结论:
\begin{quotation}
    设$K$是平面上的非空紧集.
    \begin{enumerate}[(a)]
        \item 如果$K$是连通的, 那么$\dim H^1(\mathbb{R}^2\backslash K)=1$, 且$H^1(\mathbb{R}^2\backslash K)=\left\langle[\omega_P]\right\rangle$, 对任意$P\in K$;
        \item 如果$K$不连通且$P,Q$在不同的连通分支中, 那么$[\omega_P]$与$[\omega_Q]$在$H^1(\mathbb{R}^2\backslash K)$中线性无关;
        \item 如果$K$恰好有两个连通分支, 那么任取两个连通分支中的$P,Q$, 有$H^1(\mathbb{R}^2\backslash K)=\left\langle[\omega_P],[\omega_Q]\right\rangle$.
    \end{enumerate}
\end{quotation}

习题5.13到习题5.16是证明Jordan曲线定理中处理$H^0$的技巧的一些应用.
\begin{env}{习题5.13}
    如果$A,B$是$\mathbb{R}^2$上的连通紧集, 且$A\cap B$非空而不连通, 证明$\mathbb{R}^2\backslash(A\cup B)$不连通.
\end{env}

\begin{env}{习题5.14}
    如果$Y$是平面上同胚于一个矩形或者闭圆盘的子集, 证明它的补是连通的.
\end{env}

\begin{env}{习题5.15}
    如果$X$是平面上同胚于一个圆环的子集, 证明它的补有两个连通分支.
\end{env}

\begin{env}{习题5.16}
    如果$X$是平面上同胚于一个``$8$字形''的子集, 证明它的补有三个连通分支.
\end{env}
此题略过不证.

命题5.17到推论5.19是Jordan曲线定理的一个简单应用.
\begin{env}{命题5.17}
    设$D$是闭圆盘, 记$D^\circ$为其内部, $C$为其边界圆.
    设$f:D\to\mathbb{R}^2$是连续的双射, 那么$\mathbb{R}^2\backslash f(C)$有两个连通分支, 分别为
    \[f(D^\circ)\quad\text{与}\quad\mathbb{R}^2\backslash f(D)\]
    特别地, $f(D^\circ)$是平面上的一个开集.
\end{env}

\begin{env}{推论5.18}(区域不变定理)
    如果$U$是平面上的开集, $F:U\to\mathbb{R}^2$是连续的双射, 那么$F(U)$是$\mathbb{R}^2$的开子集, 且$F$是$U$到$F(U)$的一个同胚.
\end{env}

利用区域不变定理给出第4章中的一个结论的另一证明.
\begin{env}{推论5.19}
    $2$维球面不同胚于平面的任何子集.
\end{env}

以下是一些和Jordan曲线定理相关的结论.

\begin{env}{命题5.20}
    设$F:C\to\mathbb{R}^2$是连续的双射, 那么对$\mathbb{R}^2\backslash F(C)$的有界连通分支中的任一点$P$有$W(F,P)=\pm 1$.
\end{env}

接下来两个问题是有关平面图的问题.
\begin{env}{问题5.21}(Euler公式)
    设$X$是平面上$v\geq 1$个顶点与$e\geq 0$条边构成的集合, \textit{边}指连接点与点的不自交的路径, 同时假设不同的边在顶点之外不相交.
    假设$X$有$k$个连通分支, 证明$\mathbb{R}^2\backslash X$有$f=e-v+k+1$个连通分支, 从而有
    \[v-e+f=2+(k-1)\]
\end{env}

\begin{env}{问题5.22}
    证明
    \begin{gather*}
        K_{3,3}:=(\{P_1,P_2,P_3,Q_1,Q_2,Q_3\},\{\{P_i,Q_j\}|i,j=1,2,3\})\\
        K_5:=(\{P_1,\cdots,P_5\},\{\{P_i,P_j\}|1\leq i<j\leq 5\})
    \end{gather*}
    不能被嵌入平面.
\end{env}

问题5.23是定理5.11, Jordan曲线定理以及Euler公式缩小到一个连通的开集上.
实际上证明方法是一样的, 这里不再详细写.

\begin{env}{习题5.24}
    证明定理5.11, Jordan曲线定理以及Euler公式在$2$维球面上也成立.
\end{env}

\begin{env}{问题5.25}
    在球面上找两个图, 使得他们是同胚的, 但是不存在球面之间的同胚将其中一个映为另一个.
\end{env}

\begin{env}{问题5.26}
    证明不存在M\"{o}bius带到平面的连续双射.
\end{env}

\begin{env}{问题5.27}
    设$X$是平面上同胚于圆周的子集, $P_1,P_2$在其补集中, 由一条穿过$X$ $n$次的路径连接.
    证明当$n$是偶数时$P_1,P_2$在$\mathbb{R}^2\backslash X$的同一个连通分支中, $n$为奇数时在不同的连通分支中.
    (\textit{穿过}是需要被定义的)
\end{env}

\begin{env}{问题5.28}
    一个\textbf{带边的拓扑曲面}是指一个第二可数的Hausdorff空间, 满足对任意一点$P$, 都存在$P$的一个邻域同胚于平面或上半平面$\mathbb{H}^2=\{(x,y)\in\mathbb{R}^2|\ x\geq 0\}$中的一个开集.
    如果这个同胚是到$\mathbb{H}^2$的且将$P$映为$x$轴上的点, 则称$P$为\textbf{边界点}, 否则称为\textbf{内点}\footnote{这里对带边拓扑流形的定义是更加现代的定义, 与书上略微不同}.
    证明同胚的带边曲面有着同胚的边界 (即边界点的集合), 从而M\"{o}bius带与圆锥面不同胚.
\end{env}

\printbibliography[title={参考文献}]
\end{document}