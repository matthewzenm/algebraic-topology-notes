\documentclass[11pt]{article}

\usepackage[fontset=none]{ctex}
\setmainfont{XITS}
\setCJKmainfont[BoldFont={FZCuSong-B09S},ItalicFont={FZKai-Z03S},BoldItalicFont={FZCuKaiS-R-GB}]{FZShuSong-Z01S}
\setCJKsansfont{FZHei-B01S}
\setCJKmonofont{FZFangSong-Z02S}
\usepackage[paper=a4paper]{geometry}
\usepackage{amsmath}
\usepackage{amssymb}
\usepackage{amsthm}
\usepackage{autobreak}
\usepackage{unicode-math}
\setmathfont{XITS Math}
\setmathfont{TeX Gyre Pagella Math}[range={bb,frak}]
\usepackage[shortlabels]{enumitem}
\setlist{nosep}
\usepackage{tikz-cd}
\usepackage{biblatex}
\addbibresource{biblio.bib}
\usepackage{fixdif}

\renewcommand{\proofname}{{\bf 证明}}
\theoremstyle{definition}\newtheorem*{analyse}{分析}
\theoremstyle{remark}\newtheorem{rem*}{评注}

%\def\d#1{\ \mathrm{d}#1}
\newenvironment{env}[1]{\par\vspace{1em}\noindent\textbf{#1}\ }{\par\vspace{1em}}
\DeclareMathOperator{\supp}{supp}
\DeclareMathOperator{\im}{im}

\title{de Rham上同调与Jordan曲线定理}
\author{魏子涵}
\date{\today}

\begin{document}
\maketitle

\setcounter{section}{-1}
\section{微分形式回顾}
我们回顾一下需要用到的有关微分形式的知识.

一个\textbf{$1$--形式}是一个形式和$p\d{x}+q\d{y}$, 其中$p,q$是光滑函数.
一个光滑函数$f$的\textbf{微分}是一个$1$--形式$(\partial f/\partial x)\d{x}+(\partial f/\partial y)\d{y}$.
$1$--形式中的\textbf{闭形式}是指满足$\partial p/\partial x=\partial q/\partial y$的$p\d{x}+q\d{y}$;
\textbf{恰当形式}是指作为一个光滑函数微分的$1$--形式.
闭形式与恰当形式均构成实向量空间.

一个简单的结论是
\begin{env}{准则1.10}
    恰当形式都是闭形式. \cite[p.\ 10]{Fulton1995}
\end{env}

以及一个部分的逆命题
\begin{env}{命题1.12}
    设$U$是两个有限或无限长的开区间的乘积, 那么在$U$上闭形式都是恰当形式.
\end{env}

我们之后需要一个更强的命题
\begin{env}{Poincar\'{e}引理}
    如果$U$中存在一点$P$, 使得任意$Q\in U$与$P$连接的线段均在$U$内 (即$U$是\textit{星形区域}), 那么$U$中的闭形式都是恰当形式.
\end{env}

证明参阅~\cite[定理4.11]{Spivak1965}.
此外, 我们需要一个关于卷绕数的命题
\begin{env}{命题3.16}
    设$\gamma$是闭路径, $\supp\gamma$是$\gamma$的图像,
    那么$\gamma$的卷绕数 (作为点的函数) 在$\mathbb{R}^2\backslash\supp\gamma$的每个连通分支上是常值函数.
    特别地, 在无界的连通分支上卷绕数为$0$.
\end{env}

\section{de Rham上同调群}
本书在这一章只考虑平面上一个开集$U$的第零阶和第一阶上同调群, 分别定义为
\begin{gather*}
    H^0(U)=\{f\in C^\infty(U)|\ f\ \text{局部为常值}\}\\
    H^1(U)=\frac{\{U\text{上的闭}1\text{--形式}\}}{\{U\text{上的恰当}1\text{--形式}\}}
\end{gather*}
这两者实质上是实向量空间, 但在传统上会把上同调叫做群.

容易看出$\dim H^0(U)$就是$U$的连通分支个数.
而对于$H^1$, 这一节有一个简单的计算 (命题5.1), 我们直接证明问题5.2的推广形式:
\begin{env}{问题5.2}
    设$X=\{P_1,\cdots,P_n\}$是$\mathbb{R}^2$上$n$个点构成的集合, 那么$H^1(\mathbb{R}^2\backslash X)\cong\mathbb{R}^n$.
\end{env}
\begin{analyse}
    本书是通过直接给出生成元来证明这个命题的.
    对$P=(x_0,y_0)\in\mathbb{R}^2$与某个含$P$的开集$U$, 设$1$--形式
    \[\omega_P:=\frac{1}{2\pi}\frac{-(y-y_0)\d{x}+(x-x_0)\d{y}}{(x-x_0)^2+(y-y_0)^2}\]
    那么$\omega_P$都是闭形式, 记$[\omega_P]$为$\omega_P$在$H^1(U\backslash\{P\})$中的等价类.
    我们希望证明$n$个形式的等价类$[\omega_{P_1}],\cdots,[\omega_{P_n}]$是$H^1(\mathbb{R}^2\backslash X)$的一组基.
    这需要用到引理1.17到问题1.19的一个结论:
    设$r<\min_{i,j}\{|P_iP_j|\}$, $C_i$为以$P_i$为圆心半径为$r$的圆, 如果$1$--形式$\omega$满足对每个$i=1,2,\cdots,n$都有
    \begin{equation}
        \int_{C_i}\omega=0\label{prob1.19}
    \end{equation}
    那么$\omega$在$\mathbb{R}^2\backslash X$上是恰当的.
    证明这个结论通过归纳法, 通过在一个点处画两根直线, 将平面分为$4$个半平面, 然后在这四个半平面上均存在光滑函数使其微分为$\omega$,
    积分为$0$保证可以通过调整常数使得这些函数在半平面的重合处相等, 从而$\omega$使恰当的.
\end{analyse}
\begin{env}{问题1.19}
    设$U$是两个有限或无限长的开区间的乘积, $X$是$U$中有限个点的集合, 那么满足~\eqref{prob1.19}~式假设的$1$--形式$\omega$是恰当的.
\end{env}
\vskip 8cm

然后我们给出问题5.2的证明.
\vskip 8cm

本章还会用到一个结论
\begin{env}{命题5.3}
    设$A$是$\mathbb{R}^2$上的连通闭集, $P,Q\in A$, 那么在$H^1(\mathbb{R}^2\backslash A)$中有$[\omega_P]=[\omega_Q]$.
\end{env}
\vskip 8cm

\section{上边缘映射}
我们换用一个更明晰一些的方法来讲这一节的内容.

对两个开集$U$与$V$, 设线性映射
\begin{gather*}
    i^*:H^0(U)\to H^0(U\cap V),\ f\mapsto f|_{U\cap V}\\
    j^*:H^0(V)\to H^0(U\cap V),\ g\mapsto g|_{U\cap V}\\
    k^*:H^1(U\cap V)\to H^1(U),\ [\omega]\mapsto[\omega|_U]\\
    l^*:H^1(U\cap V)\to H^1(V),\ [\tau]\mapsto[\tau|_V]
\end{gather*}
那么定义\textbf{上边缘映射}为一个线性映射$\delta:H^0(U\cap V)\to H^1(U\cup V)$, 满足序列
\begin{equation}
    H^0(U)\oplus H^0(V)\xrightarrow{i^*-j^*}H^0(U\cap V)\xrightarrow{\delta}H^1(U\cup V)\xrightarrow{k^*\oplus l^*}H^1(U)\oplus H^1(V)
    \label{mayer-vietoris}
\end{equation}
是正合的.
按照这个定义, $\delta$满足
\begin{enumerate}
    \item 对$f\in\ker\delta$, 存在$U$和$V$上的局部常值函数$f_1,f_2$使得$f=f_1-f_2$ (严格写应该是$f_1|_{U\cap V}-f_2|_{U\cap V}$, 但书上的命题5.7没有这么写);
    \item 如果$\omega\in\im\delta$, 那么$\omega$在$U$和$V$上的限制都是恰当的 (命题5.9前半部分);
    \item 如果$H^1(U)=H^1(V)=0$, 那么$\delta$是满射 (命题5.9后半部分).
\end{enumerate}

我们证明上边缘映射的存在性.
这需要用到\textit{单位分解定理}.
(证明见~\cite[附录B2]{Fulton1995}~或~\cite[pp.\ 63--64]{Spivak1965})

\vskip 12cm

{\it 上边缘映射本质上是\textbf{Mayer-Vietoris序列}的一部分, 在第10章与第23, 24章会讲到这个.}

\section{Jordan曲线定理}
这一节要证明
\begin{env}{定理5.10}(Jordan曲线定理)
    如果$X\subset\mathbb{R}^2$与一个圆周同胚, 那么$\mathbb{R}^2\backslash X$有两个连通分支, 一个有界, 另一个无界.
    $X$上任意一点的任意邻域均同时包含这两个两个连通分支中的点.
\end{env}
\noindent 而这依赖于一个命题
\begin{env}{定理5.11}
    如果$Y\subset\mathbb{R}^2$同胚于一个闭区间, 那么$\mathbb{R}^2\backslash Y$是连通的.
\end{env}
\noindent 这两个命题都需要代数拓扑的方法来证明.
\vskip 14cm

\end{document}